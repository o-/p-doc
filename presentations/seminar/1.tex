\documentclass{beamer}
\usepackage[utf8]{inputenc}
\usepackage{listings}
\usetheme{Copenhagen}
\title[Pinocchio on Speed]{Pinocchio on Speed \\ Towards a dynamic, self sustainable, native Smaltalk implementation}
\author{Oli Flückiger}
\institute{scg.unibe.ch}
\date{May 17, 2011}

\AtBeginSection[]
{
  \begin{frame}<beamer>
    \frametitle{Layout}
    \tableofcontents[currentsection,currentsubsection]
  \end{frame}
}

\begin{document}



\begin{frame}
\titlepage
\end{frame}


\begin{frame}{Solid ground?}
     \begin{center}\includegraphics[width=0.5\textwidth]{muenchhausen.png}\end{center}
\end{frame}

\section{Previous implementations}

\begin{frame}[fragile]
    \frametitle{Migrating codebase to a register VM}
    \begin{lstlisting}
void op_move {
    origin = OPERAND(2);
    target = OPERAND(1);
    STORE(target, value);
    JUMP(3);
}

    ...

program_counter = &method_code;
for (;;) {
    (*program_counter)();
}
    \end{lstlisting}
\end{frame}

\section{Removing cpu-instructions}

\begin{frame}[fragile]
    \frametitle{Threaded execution - First prototype}
    \begin{lstlisting}

void ** pc = method->code;
goto **pc;
            
    ...

load_argument:
    origin = pc[1];
    target = pc[2];
    locals[target] = arg[origin];
    goto **( pc += 3 );

    \end{lstlisting}
\end{frame}

\begin{frame}[fragile]
    \frametitle{Mapping message sends to C functions}
    \begin{lstlisting}

int mth(Method method, void** arg) {

    register void ** sp __asm("rsp");
    alloca(method->locals * sizeof(void*));

        ...

    call_method:
        callee    = pc[1];
        arguments = pc[2]; 
        mth( sp[callee], &sp[arguments] );
        goto **( pc += 3 );

}
    \end{lstlisting}
\end{frame}

\begin{frame}[fragile]
    \frametitle{Compilers and the art of persuasion}
    Problems with compiled code:
    \begin{itemize}
        \item Changing stuff is always based on guesswork what the compiler might 
                actually do with the code.
        \item Caching within opcode evaluation (e.g. self = \%rbp + 1). Therefore registers
              will be saved on {\em each} opcode evaluation
        \item Various details are crucial for speed: 
                e.g. frame pointer couldn't be ommited even tough we know the frame size.
    \end{itemize}
\end{frame}

\section{Kill the VM}

\begin{frame}[fragile]
    \frametitle{Observations}
    \begin{itemize}
        \item The VM had only few instructions
        \item Only complex opcodes were message sends and block evaluation
        \item JIT compilation as a (near) future goal
        \item So message send behaviour (e.g. stack usage) should match the cpu supported call/leave
        \item Therefore those instructions have to be implemented very specific anyway
    \end{itemize}
\end{frame}

\begin{frame}[fragile]
    \frametitle{Conclusions}
    {\bf Pinocchio self sustainable - \\Compile directly to machine code!}
    \begin{itemize}
        \item The old VM opcodes can be replaced by a few machine instructions
        \item The VM can be replaced by a compiler framework
        \item All compiler steps and all intermediate representations can 
            be implemented in a clean and polymorphic way using Smalltalk objects.
    \end{itemize}
\end{frame}

\begin{frame}[fragile]
    \frametitle{Some examples I}
    {\bf compiling a message send}
    \begin{lstlisting}

receiverTmp := receiver accept: self.
    
arguments withIndexDo: [ :arg :index |
    evaluatedArguments 
        at: index 
        put: (arg accept: self) ].
    
arguments withIndexDo: [ :arg :index |
    builder
        assign: (evaluatedArguments at: index)
        to: (self arg: index + 2 of: argsize) ].
    \end{lstlisting}
\end{frame}

\begin{frame}[fragile]
    \frametitle{Some examples I}
    {\bf compiling a message send (cont.)}
    \begin{lstlisting}
    
builder 
    assign: receiverTemp 
    to: ( self arg: 2 of: argsize ).
builder 
    indirectLoadConstant: message selector 
    into: ( self arg: 1 of: argsize ).
        
builder invoke.

ret := builder nextTemp.
builder assign: self resultVariable to: ret.

    \end{lstlisting}
\end{frame}

\begin{frame}[fragile]
    \frametitle{Some examples II}
    {\bf allocating registers}
    \begin{lstlisting}
linearScanRegisterAllocation
  liveness do: [ :interval |
    self expireIntervalsBefore: interval start.
    (active size == self numberOfRegisters)
      ifTrue: [ self spill: interval ]
      ifFalse: [ 
        self nextFreeRegisterFor: interval variable.  
        active add: interval ]].
    \end{lstlisting}
\end{frame}

\begin{frame}[fragile]
    \frametitle{Advantages}
    {\bf Play nice with the system}
    \begin{itemize}
        \item Access to C libraries provides a generic and easy interface to the operating system
        \item And: external C functions can be linked and used to implement native functions
        \item Of course in the end natives should be compiled with our own 
                compiler generating static code
        \item This compiler can be implemented directly as an extension to the current compiler
        \item The complete compiler framework is directly accessible within Pinocchio runtime
        \item JIT is just another usecase...
    \end{itemize}
\end{frame}

\section{To boldly go where...}

\end{document}
